\documentclass[11pt]{article}

%\usepackage[utf8]{inputenc}
\usepackage[a4paper, margin=1in]{geometry}


\usepackage{graphicx}
\usepackage{float}
\usepackage{xcolor}
\usepackage{enumerate}

\usepackage{amsthm}

\usepackage{natbib}

\setlength\parindent{0pt}
\setlength\parskip{5pt}

\usepackage{listings}
\lstset{
basicstyle=\small\ttfamily,
columns=flexible,
breaklines=true,,
stepnumber=1,
}

\definecolor{silver}{gray}{0.9}

\theoremstyle{definition}

\newsavebox\notebox
\newtheorem{mynote}{Note}
\newenvironment{note}%
  {\begin{lrbox}{\notebox}%
   \begin{minipage}{\dimexpr\linewidth-2\fboxsep}
   \begin{mynote}}%
  {\end{mynote}%
   \end{minipage}%
   \end{lrbox}%
   \begin{trivlist}
     \item[]\colorbox{silver}{\usebox\notebox}
   \end{trivlist}}

\newsavebox\examplebox
\newtheorem{myexample}{Example}
\newenvironment{example}%
  {\begin{lrbox}{\examplebox}%
   \begin{minipage}{\dimexpr\linewidth-2\fboxsep}
   \begin{myexample}}%
  {\end{myexample}%
   \end{minipage}%
   \end{lrbox}%
   \begin{trivlist}
     \item[]\colorbox{silver}{\usebox\examplebox}
   \end{trivlist}}


\title{The MobiView graphical user interface}
\author{Magnus Dahler Norling}

\begin{document}

\maketitle

\tableofcontents

\section{Introduction}

MobiView is a graphical user interface that can load any model built using the Mobius dll interface. This document is a work in progress.

\section{Goodness of fit statistics}

Most of the goodness-of-fit statistics are implemented following \cite{krause05}. Further properties of the various measures can be found in that paper.

Let $o=\{o_i\}_{i\in I}$ be the observed timeseries, and let $m=\{m_i\}_{i\in I}$ be the modelled timeseries. The set $I$ of comparison points is the set of all timesteps in the GOF interval where both series have a valid value. For instance, the observed timeseries can have missing values, so the timesteps corresponding to the missing values will not be considered when evaluating goodness-of-fit. The GOF interval is the entire model run interval unless something else is specified by the user. Let
\[
\overline{m} = mean(m)
\]
denote the mean of a timeseries.

\subsection{Common data points}
The common data points is the size of the set of comparison points $I$, denoted $|I|$.

\subsection{Mean error (bias)}
The mean error is
\[
\overline{o - m} = \overline{o} -\overline{m} =\frac{1}{|I|} \sum_{i\in I} (o_i - m_i)
\]
For fluxes or flows, the mean error is related to the discrepancy in mass balance.

\subsection{MAE}
MAE is the mean absolute error
\[
\frac{1}{|I|}\sum_{i\in I}|o_i - m_i|,
\]
where $|\cdot|$ denotes the absolute value of a number.

\subsection{RMSE}
RMSE is the root mean square error
\[
\sqrt{\frac{1}{|I|}\sum_{i\in I}(o_i-m_i)^2}
\]

\subsection{N-S}
N-S is the Nash-Sutcliffe efficiency coefficient
\[
1 - \frac{\sum_{i\in I}(o_i - m_i)^2}{\sum_{i\in I}(o_i-\overline{o})^2}
\]
This coefficient takes values in $(-\infty, 1]$, where a value of $1$ means a perfect fit, while a value of $0$ means that the modeled series is a no better fit than the mean of the observed series.

\subsection{log N-S}
log N-S is the same as N-S, but where $o_i$ is replaced by $\ln(o_i)$ and $m_i$ replaced by $\ln(m_i)$ for each $i\in I$, where $\ln$ denotes the natural logarithm.
\[
1 - \frac{\sum_{i\in I}(\ln(o_i) - \ln(m_i))^2}{\sum_{i\in I}(\ln(o_i)-\overline{\ln(o)})^2}
\]
This coefficient behaves similarly to N-S, but is less sensitive to errors during high values.

\subsection{r2}
$r^2$ is the coefficient of determination
\[
\left(\frac{\sum_{i\in I}(o_i-\overline{o})(m_i-\overline{m})}{\sqrt{\sum_{i\in I}(o_i-\overline{o})^2}\sqrt{\sum_{i\in I}(m_i-\overline{m})^2}}\right)^2
\]
This coefficient takes values in $[0, 1]$.

\subsection{Idx. of agr.}
The index of agreement is
\[
1 - \frac{\sum_{i\in I}(o_i-m_i)^2}{\sum_{i\in I}(|m_i-\overline{o}| + |o_i-\overline{o}|)^2}
\]

\subsection{Spearman's RCC}
Spearman's rank correlation coefficient \cite{spearman04} is computed as follows: For a timeseries $x={x_i}_{i\in_I}$, let $\mathrm{rank}(x_i)$ be the index of $x_i$ (starting from 1) in the list $\mathrm{sort}(x)$, where $\mathrm{sort}(x)$ is $x$ sorted from smallest to largest. The rank correlation coefficient can then be computed as
\[
1 - \frac{6\sum_{i\in I}(\mathrm{rank}(o_i)-\mathrm{rank}(m_i))^2}{|I|(|I|^2 - 1)}
\]
The coefficient takes values in $[-1, 1]$. If the value is 1, the modeled series is a (positively) monotone function of the observed series.


\bibliographystyle{plain}
\bibliography{citations}

\end{document}