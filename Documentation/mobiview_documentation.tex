\documentclass[11pt]{article}

%\usepackage[utf8]{inputenc}
\usepackage[a4paper, margin=1in]{geometry}


\usepackage{graphicx}
\usepackage{float}
\usepackage{xcolor}
\usepackage{enumerate}

\usepackage{amsthm}

\usepackage{natbib}

\usepackage{hyperref}
\usepackage{graphicx}

\usepackage[font=small,labelfont=bf]{caption}

\setlength\parindent{0pt}
\setlength\parskip{5pt}

\usepackage{listings}
\lstset{
basicstyle=\small\ttfamily,
columns=flexible,
breaklines=true,,
stepnumber=1,
}

\definecolor{silver}{gray}{0.9}

\theoremstyle{definition}

\newsavebox\notebox
\newtheorem{mynote}{Note}
\newenvironment{note}%
  {\begin{lrbox}{\notebox}%
   \begin{minipage}{\dimexpr\linewidth-2\fboxsep}
   \begin{mynote}}%
  {\end{mynote}%
   \end{minipage}%
   \end{lrbox}%
   \begin{trivlist}
     \item[]\colorbox{silver}{\usebox\notebox}
   \end{trivlist}}

\newsavebox\examplebox
\newtheorem{myexample}{Example}
\newenvironment{example}%
  {\begin{lrbox}{\examplebox}%
   \begin{minipage}{\dimexpr\linewidth-2\fboxsep}
   \begin{myexample}}%
  {\end{myexample}%
   \end{minipage}%
   \end{lrbox}%
   \begin{trivlist}
     \item[]\colorbox{silver}{\usebox\examplebox}
   \end{trivlist}}


\title{The MobiView graphical user interface}
\author{Magnus Dahler Norling}

\begin{document}

\maketitle

\tableofcontents

\section{Introduction}

MobiView is a graphical user interface that can load any model built using the Mobius dll interface. It is created using the Ultimate++ framework and the ScatterCtrl package (\url{ultimatepp.org}).

Note: at any time there may have been updates to MobiView that are not yet covered in this document, but we try to be as diligent as possible.

\begin{center}
\includegraphics[width=\linewidth]{img/mobiview}
\captionof{figure}{Overview of the MobiView user interface}
\end{center}

\section{The toolbar}

The toolbar has the following buttons.

\begin{center}
\includegraphics[width=0.5\linewidth]{img/toolbar}
%\captionof{figure}{The toolbar}
\end{center}

\begin{enumerate}
\item Load (ctrl-O). This asks you to select a model dll, an input file, and a parameter file. (On Linux, you load a shared object .so file instead of a dll). The input and parameter files have to be of the .dat format described in the Mobius file format documentation. You can click load even if you have a model and dataset loaded already. This will delete all data from the earlier dataset from memory, so be sure to save any changes you want to keep first.
\item Save parameters (ctrl-S). This saves any edits you have made to the parameters to the current working parameter file (usually the one you loaded).
\item Save parameters as (alt-S). Saves the parameters to a new file. The new file is now the current working file.
\item Search parameters (ctrl-F). Opens a new window that allows you to search for parameters by name. Any matching parameters are displayed in a list, and clicking an item in the list takes you to the right parameter group in the main view. The search is a case-insensitive substring search.
\item Edit indexes. Opens a new window. This is described in Section \ref{sec:editindexes}.

\item Run model (F7). Runs the model using the parameters that are loaded in the MobiView interface, taking into account any edits. (It does not matter if the edits have been saved to file or not). Results from the latest model run are available for plotting. See section \ref{sec:plotting}
\item Open an additional plot view to display multiple plots at the same time. This is further explained in section \ref{sec:additionalplotview}.
\item Export to csv (ctrl-E). Allows you to save all selected time series in a .csv format. The separator is ;

\item Save baseline. Saves a background copy of the current dataset. This is used in the "Compare baseline" plot mode. See Section \ref{sec:comparebaseline}.
\item Revert to baseline. Restores the baseline parameters and result series to the main dataset if a baseline was saved earlier.
\item Perturbation view. Allows you to see the effect on one time series of perturbing a single parameter. See Section \ref{sec:sensitivity}
\item Optimization view. Allows you to autocalibrate the model. Documentation of this feature is forthcoming. See Section \ref{sec:optimization}

\item Edit statistics settings. Here you can turn on or off what statistics you want displayed in the timeseries info box, and also choose percentiles for the Q-Q plot mode.
\item View model equation batch structure. This displays the equation batch structure, as described in the Mobius model builder documentation. This is mostly interesting if you are a model developer, as it can provide some debug information for your work-in-progress model.
\item View model info. This opens up a window with some information about each module in the current model, with links to further references.
\end{enumerate}

\section{Parameter viewing and editing}

\begin{center}
\includegraphics[width=\linewidth]{img/paramedit}
\captionof{figure}{Parameter display and editing}
\end{center}

\subsection{Setting up a new project}

One efficient way of setting up a new project is taking an existing project and using the index set editor described in Section \ref{sec:editindexes}.

Another way is to take the parameter file of an existing project and delete everything below the "parameters:" section. You can then fill in the indexes you want in the index sets (see the file format documentation in a separate document). This file can be loaded in MobiView. MobiView does not mind that not all parameters are given values in the parameter file, and will assume default values for the ones that are missing. If you save the file again after having loaded it, all the new values will be filled in in the file.

\subsection{The parameter group view and parameter index selection}

In the parameter group view you can select what parameter group to view the parameters of. These are usually organized under what module (i.e. sub-model) they belong to. Each parameter group indexes zero or more index sets. In the parameter index selection menus you can choose the (tuple of) indexes that you want for the current the parameter view display. Only the index sets that the currently selected group indexes over are active, the menus of the other index sets are disabled.

If you check the 'Lock' box under one or more of the index sets, any edit to a parameter value will be performed to all value instances over those index sets, not just to the value corresponding to the current index tuple.

If you check the 'Expand' button under one index set, the parameter view will gain one additional column for that index set, and will display all parameter values over that index set.

\subsection{The parameter view}

The parameter view displays the full name, the value, the recommended min and max values, the unit and the description of each parameter in the selected group. Min and max values and a description are only optionally provided by the model creator.

The value field is editable. What type of field it is depends on the type of the parameter value. For instance, a parameter of type double (double precision floating point number) is edited using a text field, while a parameter of type bool is edited using a check box since it is essentially just an on-off switch.

If a model has a parameter group where the two last index set dependencies are the same index set, the parameter view works a little differently. The parameter index selection for that index set will only apply to the first instance of the dependency. The second instance of the dependency is the row in the parameter view (so you can edit all values of that row in the same view). This does for instance apply to the percolation matrix in the PERSiST model, where you choose what row you are on using the Soil index set selection menu, and then all columns for that row are displayed in the parameter view. Note that this is so far a little limited. It does only work for parameters of type double, and it only works if it is the last two index set dependencies that are the same. If you have a different configuration in a Mobius model, it may not be that easy to use the MobiView interface with it, but this is a very small edge case that could be fixed later.

\subsection{Index set editing}\label{sec:editindexes}

\begin{center}
\includegraphics[width=0.7\linewidth]{img/editindexes}
\captionof{figure}{The index set editing window}
\end{center}

This is done by opening a separate window from the toolbar. You can edit what indexes are in each index set. For non-branched index sets, each index is a line in the associated text field at the bottom of the window. You can add, delete or rename indexes by editing the text. Note that if you add a new index or rename one, the parameters indexed by that index will be given default values.

You can also edit the branch connections of branched index sets. This is done in the the list in the top right. The branch connections are rendered in the top left. The input ids of one index (typically a reach) have to be a comma-separated list of id's of other indexes (the id is just the value displayed in the "Id" column). No information about actual length or position is used in the visualization of the connections, only the connectivity.

\section{The plot} \label{sec:plotting}

You can use the plot to visualize inputs and results of the model. Time series can be selected in the Equation and Input lists to the right of the plot (results can only be selected if the model has been run at least once). You can select multiple time series at one time by ctrl-clicking (or shift-clicking) them. You can also remove a selected time series by ctrl-clicking it. If a time series varies over one or more index sets, you can select indexes from the lists below the plot. You can do multiselection here too.

The time series info box will display info about the selected time series. If you are in a residual-type plot mode, it will also display goodness-of-fit statistics (see section \ref{sec:gof}).

The plot will automatically update itself every time you run the model to reflect any changes in the result time series.

\subsection{Navigation}
You can
\begin{enumerate}
\item Pan the plot horizontally by either holding down ctrl or the central mouse button (wheel), and moving the mouse left to right.
\item Zoom the x-axis by using the scroll wheel on the mouse or clicking ctrl+ or ctrl- on the keyboard.
\item Read the values of a given point or distance between two points by left clicking the mouse (and moving it).
\end{enumerate}

\subsection{Plot options}
You can choose between several different plot options. First, you can choose a few different plot modes. These are described in the next subsection. Depending on the plot mode, you may also have other options available

\begin{enumerate}[i]
\item Aggregation. You can choose to aggregate the displayed time series on monthly or yearly steps. You can also choose between "mean", "sum", "min", or "max" as the aggregation mode. Note that the sum aggregation only works well if all timesteps have a value (which may not be the case for some input series). For models with other than daily timesteps, monthly or yearly aggregation may not always make sense. This is a work in progress.
\item Y axis transformation. You can choose between three Y axis transformations
\begin{enumerate}
\item No transformation (regular Y axis).
\item Normalize it. In this case, every displayed time series is normalized separately so that it takes values between 0 and 1. This is useful if you want to compare the shape of time series with very different scales.
\item Logarithmic Y axis. This changes the scale of the axis to be a base 10 logarithm.
\end{enumerate}
\item Scatter inputs. Determines if input time series should be displayed as scatter plots or line plots. Scatter plots work better if there are a lot of holes in the input data.
\end{enumerate}

\subsection{Plot modes}

\subsubsection{Regular}

The Regular plot mode will just plot all selected time series as a function of time. Result time series are plotted as line plots. Input time series can be plotted as scatter plots if the Scatter inputs option is selected. This mode also works together with all other plot options.

\subsubsection{Stacked and Stacked share}

The stacked plot modes function like the regular plot mode, but makes a stacked plot instead of a line plot. Only result time series will be stacked, not input time series. The "Stacked share" mode normalizes all the time series so that they sum to 100. This allows you to visualize how large a percentage each time series is of the total.

\subsubsection{Histogram}

This option only works if you have exactly one time series selected. it will make a histogram of the time series. The number of bins is determined by %Sturges' rule \cite{sturges26}
Rice's rule %Can't find the original reference for Rice's rule :(
\[
%k = 1 + \lceil \log_2 n\rceil, 
k = 2\lceil\sqrt[3]{n}\rceil,
\]
where $n$ is the number of data points and $k$ is the number of bins. The Y axis displays the fraction of the total amount of data points that fall inside each bin.

\subsubsection{Profile}

Select one result or input. Moreover, select two or more indexes of exactly one index set that this time series varies over. For a given point in time, a bar plot will be displayed with the selected indexes as the X axis. The point in time can be selected using a slider or a text field. This mode also works together with aggregations, so you can display e.g. yearly mean values (for a selected year).

One use case for this is e.g. to select the yearly sum of the value "Reach nitrate output" in INCA-N to show a bar plot of the yearly output of each reach in a given year.

\begin{center}
\includegraphics[width=0.7\linewidth]{img/scr2}
\captionof{figure}{When using the Profile plot mode you can select the date using a slider or a date text field.}
\end{center}

\subsubsection{Compare baseline}\label{sec:comparebaseline}

This is only available if you have clicked the "Save baseline" button in the toolbar. You must have only one result time series (and optionally one input time series) selected. The plot will display both the current value of the selected time series and the value of the time series at the point you clicked "Save baseline". All plot options are available.

This can be useful for exploring differences in outcome between different parameter sets. For instance, you can see how it affects the stream nitrate concentration if you change the agricultural fertilizer nitrate input in INCA-N.

\subsubsection{Residuals}

You must have exactly one result time series and one input time series selected. The plots shows the residual time series (observed - modeled). Select "Scatter inputs" to display it as a scatter plot instead of a line plot.

To compare your modeled time series against an observed series, you can load the observed series in as an "additional timeseries", which is explained in the Mobius file format documentation.

A linear regression line of the residuals is also displayed. This shows the trend in the residuals. For instance if this trend goes up, it says that the observed quantity increases over time compared to the modeled one. The regression line is only computed for the stat interval (see section \ref{sec:gofint}).

You can also use aggregation. For instance, the monthly sum of the residuals of something like "Reach flow" versus "Observed discharge" can tell you something about the monthly water balance in a hydrology model.

\subsubsection{Residual histogram}

You must have exactly one result time series and one input time series selected. The shows a histogram of the residuals. The number of bins are selected using the same rule as for the Histogram option. Moreover, red dots show what the distribution of residuals would look like if it was perfectly normally distributed (with the same mean and standard deviation).

\subsubsection{Q-Q}

You must have exactly one result time series and one input time series selected. This shows a quantile-quantile plot of the two time series, and can be used to see if your modeled time series is roughly similarly distributed to the observed one.

The displayed percentiles can be edited in the statistics settings window, which can be opened from the toolbar. The X axis is the result series, while the Y axis is the input series. The two have the same quantiles if the blue dots are on the red diagonal line.

\subsection{Context menu options (edit or save plot)}

In addition to what we have implemented in MobiView (described above), the plot has all of the functionality of the ScatterCtrl package from the Ultimate++ framework. You can right click the plot to get a context menu, where you can
\begin{enumerate}[i]
\item Select zooming or panning options.
\item Edit text fields, select colors and plot styles (Properties).
\item Get a table of the underlying data of the plot (Data). Can e.g. be copied to Excel by selecting multiple cells and ctrl-C.
\item Copy the plot as an image to the clipboard (Copy image). This only works correctly if you want to paste into certain applications. It works with most image editing software, Microsoft applications (e.g. Word, Outlook, Internet Explorer), but sadly there is a bug right now where it does not work with the Google Chrome browser.
\item Save the image of the plot (Save image). Several formats are available (e.g. png, pdf).
\end{enumerate}

\subsection{The stat interval}\label{sec:gofint}

You can select the stat interval below the timseries info box. The interval consists of two dates, and only timesteps between these dates will be considered in the statistics and goodness-of-fit computations (see next section).

\begin{center}
\includegraphics[width=0.3\linewidth]{img/gof}
\captionof{figure}{When in a Residual-type plot mode, the stat interval can be chosen using the two date text fields below the time series info box, where the goodness-of-fit stats are displayed.}
\end{center}

\subsection{Goodness-of-fit statistics}\label{sec:gof}

The goodness-of-fit statistics are displayed in the time series info box if you have exactly one result series and one input series selected, and the "Show GOF" box is checked.

It can be useful sometimes to not have all statistics displayed. In that case, you can turn them off in the statistics settings window, opened from the toolbar.

During model calibration you can also see the changes in each statistic between runs. This is displayed to the right of the value, and is color coded to show if the change was good or bad. If the plot is in "compare baseline" mode, the changes in the GOF will be computed relative to the baseline, otherwise it is always relative to the previous run of the model.

Most of the goodness-of-fit statistics are implemented following \cite{krause05}. Further properties of the various statistics are discussed in that paper.

Let $o=\{o_i\}_{i\in I}$ be the observed time series, and let $m=\{m_i\}_{i\in I}$ be the modelled time series. The set $I$ of comparison points is the set of all timesteps in the stat interval (see Section \ref{sec:gofint} where both series have a valid value. For instance, the observed time series can have missing values, so the timesteps corresponding to the missing values will not be considered when evaluating goodness-of-fit. The stat interval is the entire model run interval unless something else is specified by the user. Let
\[
\overline{m} = \frac{1}{|I|}\sum_{i\in I}m_i
\]
denote the mean of a time series.

\subsubsection{Common data points}
The common data points is the size of the set of comparison points $I$, denoted $|I|$.

\subsubsection{Mean error (bias)}
The mean error is
\[
\overline{o - m} = \overline{o} -\overline{m} =\frac{1}{|I|} \sum_{i\in I} (o_i - m_i)
\]
For fluxes or flows, the mean error is related to the discrepancy in mass balance.

\subsubsection{MAE}
MAE is the mean absolute error
\[
\frac{1}{|I|}\sum_{i\in I}|o_i - m_i|,
\]
where $|\cdot|$ denotes the absolute value of a number.

\subsubsection{RMSE}
RMSE is the root mean square error
\[
\sqrt{\frac{1}{|I|}\sum_{i\in I}(o_i-m_i)^2}.
\]

\subsubsection{N-S}
N-S is the Nash-Sutcliffe efficiency coefficient \cite{nashsutcliffe70}
\[
1 - \frac{\sum_{i\in I}(o_i - m_i)^2}{\sum_{i\in I}(o_i-\overline{o})^2}.
\]
This coefficient takes values in $(-\infty, 1]$, where a value of $1$ means a perfect fit, while a value of $0$ or less means that the modeled series is a no better fit than a series constantly equal to the mean of the observed series.

\subsubsection{log N-S}
log N-S is the same as N-S, but where $o_i$ is replaced by $\ln(o_i)$ and $m_i$ replaced by $\ln(m_i)$ for each $i\in I$. Here $\ln$ denotes the natural logarithm.
\[
1 - \frac{\sum_{i\in I}(\ln(o_i) - \ln(m_i))^2}{\sum_{i\in I}(\ln(o_i)-\overline{\ln(o)})^2}.
\]
This coefficient behaves similarly to N-S, but is less sensitive to errors on time steps where both series have large values.

\subsubsection{r2}
$r^2$ is the coefficient of determination
\[
\left(\frac{\sum_{i\in I}(o_i-\overline{o})(m_i-\overline{m})}{\sqrt{\sum_{i\in I}(o_i-\overline{o})^2}\sqrt{\sum_{i\in I}(m_i-\overline{m})^2}}\right)^2.
\]
This coefficient takes values in $[0, 1]$.

\subsubsection{Idx. of agr.}
The index of agreement is
\[
1 - \frac{\sum_{i\in I}(o_i-m_i)^2}{\sum_{i\in I}(|m_i-\overline{o}| + |o_i-\overline{o}|)^2}.
\]

\subsubsection{KGE}
KGE is the Kling-Gupta efficiency \cite{klinggupta09}
\[
1 - \sqrt{(r-1)^2 + (\beta-1)^2 + (\delta-1)^2}
\]
where $r$ is the square root of the coefficient of determination $r^2$, $\beta=\overline{m}/\overline{o}$, and $\delta=Cv(m)/Cv(o)$, $Cv(x)=\sigma(x)/\overline{x}$, $\sigma$ being the standard deviation.

\subsubsection{Spearman's RCC}
Spearman's rank correlation coefficient \cite{spearman04} is computed as follows: For a time series $x=\{x_i\}_{i\in_I}$, let $\mathrm{rank}(x_i)$ be the index of $x_i$ (starting from 1) in the list $\mathrm{sort}(x)$, where $\mathrm{sort}(x)$ is $x$ sorted from smallest to largest. The rank correlation coefficient can then be computed as
\[
1 - \frac{6\sum_{i\in I}(\mathrm{rank}(o_i)-\mathrm{rank}(m_i))^2}{|I|(|I|^2 - 1)}.
\]
The coefficient takes values in $[-1, 1]$. If the value is 1, the modeled series is a (positively) monotone function of the observed series.

\subsection{Hydrological indexes}

In addition to the common statistics like mean, standard deviation etc., MobiView computes some hydrological indexes. These are mostly meaningful only for flow series, but are computed for all series any way. See e.g. \cite{fenicia-et-al18} for reference.

\subsubsection{Flashiness}

The flashiness index is defined as
\[
\frac{\sum_i=2^N}|m_i-m_{i-1}|/\overline{m}
\]
and says something about how quick the time series is to respond to events.

\subsubsection{Baseflow index (filter approximation)}

The baseflow index approximation (est.bfi) is
\[
\frac{\sum{m_i^b}}{\sum{m_i}}
\]
where $m_i^b$ is the result of applying the single pass forward filter
\[
m_i^b = \min{m_i, am_{i-1}^b + (1-a)\frac{m_{i-1}+m_i}{2}}
\]
The filter constant $a$ can be set in the statistics settings menu.

This index is an approximation that tries to separate out how much of the signal comes from a long-frequency source.

Note that even if you have a model that uses a "Baseflow index" parameter, this approximation will probably not coincide with that parameter, but should be linearly correlated with it.

\subsection{The additional plot view}\label{sec:additionalplotview}

\begin{center}
\includegraphics[width=\linewidth]{img/additionalplotview}
\captionof{figure}{The additional plot view}
\end{center}

From the toolbar you can open an additional window to display multiple plots at the same time.

In the top you can select how many plots are to be shown ("Rows"), and if the x axes of the plots are to be linked.

For each row you can either push the "$<<$" button to copy the plot setup from the main window to the local one or push the " $>>$" button to copy the local plot setup back to the main window. Like the main plot, these also update whenever you run the model.

This window has a toolbar with two buttons on its own
\begin{enumerate}
\item Save. You can store the current plot setup to a file.
\item Load. You can load the current plot setup from a saved file. Note that the actual time series data will be from the current model run, not the one that was active when the setup was stored. Only the actual plot setup (i.e. name of time series, indexes, plot mode etc.) is stored and loaded.
\end{enumerate}

\section{Sensitivity and optimization}

\subsection{Single parameter perturbation}\label{sec:sensitivity}

\begin{center}
\includegraphics[width=\linewidth]{img/parameterperturbation}
\captionof{figure}{Parameter perturbation analysis}
\end{center}

From the toolbar you can open a single-parameter perturbation analysis window. You must select a single result series in the main window. Optionally you can also select a single input series. You can only have one index selected per index set in the time series index selection.

You must also select a parameter in the parameter view of the main window. Next, in the perturbation view window, you can select the minimum and maximum of the perturbation range, and select how many model evaluations will be done along that range. The steps along the parameter value will be equally spaced. All other parameters are held constant. This uses a copy of a the main dataset, and does not affect the value of this parameter in your main dataset. The plot will show one time series for each value of the perturbed parameter.

If an index set was checked as "locked" in the main window, all values across the indexes of that index set will be treated as a single parameter, and all of them will be given the same value.

You can also select a statistic to plot in the rightmost plot, with the perturbed parameter as the x axis and the statistic as the y axis. If you want to plot a goodness-of-fit statistic (these are documented in section \ref{sec:gof}, you must have an input series selected that will be used as the comparison series. The GOF-interval from the main window will be used (section \ref{sec:gofint}, you need to be in residual plot mode in the main window to access it for editing).


\subsection{Optimization and Monte-Carlo sensitivity and uncertainty analysis}\label{sec:optimization}

\begin{center}
\includegraphics[width=\linewidth]{img/mobiview_optimizer}
\captionof{figure}{Optimization and MCMC setup}
\end{center}

The optimizer, or auto-calibrator, allows you to tell MobiView to search a subset of the parameter space in order to optimize a set of selected statistics. MCMC and variance based sensitivity analysis allow you to sample the given parameter space in order to determine sensitivity of model outcomes to changes in these parameters and to sample the posterior distribution of likelihoods in error structures. They all use a common parameter space and target setup.

If you select a parameter in the main window, you can click "Add parameter" in the setup to add it to the list. You can also set the min and max values that you want to constrain this parameter with. Note that all values in the range have to be valid values that don't make the model crash. For the optimizer it helps if you can narrow down the range as much as possible before running it.

You can click "Add group" to add all the parameters that are visible in the main parameter view.

If an index set was checked as "locked" in the main window at the time a parameter was added, all values across the indexes of that index set will be treated as a single parameter, and all of them will be given the same value every time the model is run. Documentation of the feature that allows using expressions to make relationships between parameters is forthcoming.

To add a target statistic, select one result series and up to one comparison input series in the main window, then click "Add target". You can select the target statistic from a list of statistics. The available statistic will depend on what kind of run you want (optimizer, MCMC or variance-based sensitivity). The total statistic that is considered in each model run is the weighted sum of the statistics for the individual targets. All statistics are computed inside the stat interval only (see Section \ref{sec:gofint}.

If you need a more complex setup than the one you can make here, we recommend that you use the Mobius python wrapper to script it instead.

In the toolbar of the optimization setup window you can click buttons to save the current setup for future use, or load it again.

\subsubsection{optimizer}
All the GOF statistics (section \ref{sec:gof}) are available for optimization. The statistics MAE and RMSE will be minimized, the others maximized. You can not mix targets that should be minimized with targets that should be maximized. All targets have to have an input comparison series for it to be able to evaluate the error of the modeled series compared to the observed series.

You can set the maximal number of times the optimizer is allowed to evaluate the model. It will typically use all of these. The higher number, the better result, but also longer run time. The total time of the optimizer is roughly equal to the time to run one model multiplied with the number of evaluations (overhead of the algorithm is small in comparison).

The main parameter set (that is loaded in memory) is replaced with the new best parameter set when the optimization run is finished.

We use the dlib global optimization algorithm from \url{http://dlib.net/optimization.html}.

If you have the "show progress" box ticked, it will periodically show how many evaluations it has done and what is the best current value of the (compound) target statistic compared to the initial value of the statistic. If you additionally have the additional plot view open with the targeted time series

\subsubsection{MCMC}
Documentation of this feature is forthcoming

\subsubsection{Variance based sensitivity}
We compute two statistics that say something about how much of the variance of the (compound) target statistic (over the given parameter space) is accounted for by each individual parameter.

We compute the first-order sensitivity coefficient (main effect index) and total effect index of the listed parameters on the chosen (compound) target statistic. The algorithms used for computation are based on the ones suggested in \cite{saltelli-et-al10}.

The first-order sensitivity coefficient is
\[
S_i = \frac{V_{X_i}(E_{X_{\sim i}}(Y|X_i))}{V(Y)}
\]
where $V$ is the variance, $E$ the expected value, $X=(X_i)_{i=1}^n$ the parameter vector, $X_{\sim i}$ the parameter space excluding parameter $i$, and $Y$ the (compound) target statistic.

The total effect index is
\[
S_{T_i} = \frac{E_{X_{\sim i}}(V_{X_i}(Y|X_{\sim i}))}{V(Y)}
\]
The total effect index measures both the first-order effect and the higher-order effects (interactions between parameter $i$ and others). In one has
\[
\sum_{i=1}^n S_i \leq 1,\; \sum_{i=1}^n S_{T_i} \geq 1.
\]

Note that since these are estimated using a Monte-Carlo method, they will not be exact, and especially small values are unreliable. The first-order index can be shown as negative even though the exact value has to be positive, and the first order index could be shown as smaller than the total index even though that is not possible for the exact values. You can try to run with a higher number of samples if this becomes a problem.

\section{Known issues}

Windows 10 may scale the layout of your programs, and for MobiView this can make the window too large for your screen if the screen resolution is not very high. If this happens:
\begin{itemize}
\item Right click your desktop, choose 'display settings', then go to 'Change the size of text, apps, and other items' and set it to 100\%.
\end{itemize}
If this does not work right away, you could try to
\begin{itemize}
\item Restart Windows.
\item Right click MobiView.exe, then go to Settings$\rightarrow$Compatibility, click 'Change high DPI settings', then check the first box under 'Program DPI', then change 'Use the DPI that's set for my main display when' to 'I open this program'.
\end{itemize}
This is not an ideal solution since it also changes the scaling of all your other programs and may make text hard to read. We will try to see if there is a better solution.

\bibliographystyle{plain}
\bibliography{citations}

\end{document}