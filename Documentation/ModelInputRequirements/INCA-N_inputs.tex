\documentclass[11pt]{article}

%\usepackage[utf8]{inputenc}
\usepackage[a4paper, margin=1in]{geometry}


\usepackage{graphicx}
\usepackage{float}
\usepackage{xcolor}
\usepackage{enumerate}
\usepackage{gensymb}

\usepackage{amsthm}

\usepackage{natbib}

\setlength\parindent{0pt}
\setlength\parskip{5pt}

\usepackage{listings}
\lstset{
basicstyle=\small\ttfamily,
columns=flexible,
breaklines=true,,
stepnumber=1,
}

\definecolor{silver}{gray}{0.9}

\theoremstyle{definition}

\newsavebox\notebox
\newtheorem{mynote}{Note}
\newenvironment{note}%
  {\begin{lrbox}{\notebox}%
   \begin{minipage}{\dimexpr\linewidth-2\fboxsep}
   \begin{mynote}}%
  {\end{mynote}%
   \end{minipage}%
   \end{lrbox}%
   \begin{trivlist}
     \item[]\colorbox{silver}{\usebox\notebox}
   \end{trivlist}}

\newsavebox\examplebox
\newtheorem{myexample}{Example}
\newenvironment{example}%
  {\begin{lrbox}{\examplebox}%
   \begin{minipage}{\dimexpr\linewidth-2\fboxsep}
   \begin{myexample}}%
  {\end{myexample}%
   \end{minipage}%
   \end{lrbox}%
   \begin{trivlist}
     \item[]\colorbox{silver}{\usebox\examplebox}
   \end{trivlist}}


\title{Inputs to INCA-N}
\author{Magnus Dahler Norling}

\begin{document}

\maketitle

%\tableofcontents

\section{Introduction}

This document describes what inputs are used by the INCA-N model. For documentation of the model itself, see the published paper \cite{wade02}, however note that the model has changed a bit since that paper. For the format of the inputs, see the file format documentation.

\begin{note}
INCA-N builds on PERSiST, and so has all the same input requirements and options as PERSiST. This document will only list the ones that are added on top of that by INCA-N.
\end{note}

\section{Required inputs}

None apart from the ones in PERSiST.

\section{Optional inputs}

If an optional input is provided at all it has to have to have a value for every day of the period you intend to run the model. If it is not included, a value is typically computed for it instead based on parameter values. If you give an optional input an index set dependency, it is fine if you only provide a timeseries for some of the indexes. The value for the other indexes will then be computed as normal.

\begin{enumerate}[i]
\item {\bf\tt "Fertilizer nitrate"}. How much NO3 is added by fertilization, in $kg/Ha/day$. Replaces the computed fertilisation. Recommended index set dependencies: {\tt \{"Landscape units"\}} or {\tt \{"Reaches" "Landscape units"\}}.
\item {\bf\tt "Fertilizer ammonium"}. How much NH4 is added by fertilization, in $kg/Ha/day$. Replaces the computed fertilisation. Recommended index set dependencies: {\tt \{"Landscape units"\}} or {\tt \{"Reaches" "Landscape units"\}}.
\item {\bf\tt "Nitrate dry deposition"}. In $kg/Ha/day$. Replaces the parameter of the same name.
\item {\bf\tt "Ammonium dry deposition"}. In $kg/Ha/day$. Replaces the parameter of the same name.
\item {\bf\tt "Nitrate wet deposition"}. In $kg/Ha/day$. Gives the actual wet deposition to the soil instead of having it computed based on the parameter and the precipitation.
\item {\bf\tt "Ammonium wet deposition"}. In $kg/Ha/day$. Gives the actual wet deposition to the soil instead of having it computed based on the parameter and the precipitation.
\item {\bf\tt "Effluent nitrate concentration"}. In $mg/l$. Replaces the parameter "Reach effluent nitrate concentration".
\item {\bf\tt "Effluent ammonium concentration"}. In $mg/l$. Replaces the parameter "Reach effluent ammonium concentration".
\end{enumerate}

\begin{note}
It may be convenient to use the format of constant periods to specify e.g. fertilization inputs. However, note that using this format will set any days you don't provide data for to NaN, which will crash the model. If you want the missing values to be 0, you should instead clear the entire input series to 0 before providing the other periods like seen below.
\begin{lstlisting}
"Fertilizer nitrate" :
"2004-01-01" to "2011-12-31" 0   #NOTE: Clears the entire series first

#everything below overwrites the above 0's.
"2004-04-10" to "2004-06-01" 0.01
<..etc..>
end_timeseries
\end{lstlisting}
\end{note}

There is also the possibility to specify multiple plant growth periods (which will influence plant ammonium uptake). This is done using the optional timeseries {\tt\bf "Growth curve offset"} and {\tt\bf "Growth curve amplitude"}. They should be specified using the constant periods format as in the next example. It is important that the periods in these two timeseries match up. Unlike with the other timeseries, the model expects the value between periods to be NaN, and uses this to determine whether a period started or ended. This replaces the parameter-based growth periods.

\begin{example}
\begin{lstlisting}
"Growth curve offset" :
"2004-04-10" to "2004-06-01" 0.33
"2004-03-27" to "2004-06-15" 0.31

"Growth curve amplitude" :
"2004-04-10" to "2004-06-01" 0.67
"2004-03-27" to "2004-06-15" 0.69
\end{lstlisting}
\end{example}

As with any other optional timeseries, you can give these index set dependencies, but it is important that "Growth curve offset" and "Growth curve amplitude" are given the same dependencies and that any pair of these with the same indexes match up. If series are not provided for some indexes, the growth period will be based on the parameters for those indexes as usual.

\bibliographystyle{plain}
\bibliography{../citations}

\end{document}


